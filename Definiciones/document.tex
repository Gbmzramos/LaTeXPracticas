\documentclass[11pt]{article}
\usepackage[utf8]{inputenc}
\usepackage{amsmath} 
\usepackage{xcolor}
\usepackage{bookman}
\usepackage{fourier}
\usepackage[medium,scaled=0.9]{FiraMono}
\usepackage[T1]{fontenc}
\usepackage{xparse}


\def\hola#1;{este es mi comando #1} 
\def\division#1,#2:{\left(#1 \mathbin{/} #2\right)}
\def\multiplica#1,#2:{\left[#1\times #2\right]} 

\def\proposicion#1;#2:{#1, \mbox{ } #2} 

\def\integral#1,#2,#3;#4:{\int_{#1} ^{#2} #3 \,  d#4} 

% Start the document
\begin{document}
\fontfamily{pbk}\selectfont

% Create a new 1st level heading
\section{\textsf{Heading}}

\textsf{Your text \textsc{isótopos cd-rom} here…}

% Uncomment the following two lines if you want to have a bibliography
%\bibliographystyle{alpha}
%\bibliography{document}

\hola \hola nombre; nuevo nombre; $f(x) = ay^2$

\[ 
 A = \int \division 12,\division a, b:: \, dx
\]

\[
\multiplica 3, \multiplica 1,a+b::
\] 
\[
\integral a, b, \integral c,d, f(x) ;x: ; y:
\] 


Comparemos codigo 
\begin{verbatim} 
\def\integral#1,#2,#3;#4:{\int_{#1}^{#2} #3 \, d{#4}} 
\[
\integral - \infty , 0, \exp{x^2};x:
\] 
\end{verbatim} 

Produce:
\[
\integral - \infty , 0, \exp{ x^2};x:
\] 

Con 
\begin{verbatim}\black!50
\[
\int_{- \infty}^0 \exp{x^2}\, dx
\] 
\end{verbatim} 

Produce:
\[
\int_{- \infty}^0 \exp{ x^2}\, dx
\]

Con 
{
\color{black!90}
\begin{verbatim} 
\newcommand\Integral[4][x]{%
\int_{#2}^{#3} #4 \, d#1%
} 
\[
\Integral[y]{-\infty}{0}{\exp{y^2}} 
\] 
\end{verbatim} }

Produce:
\newcommand{\Integral}[4][x]{%
\int_{#2}^{#3} #4 \, d#1% 
} 
\[
\Integral[y]{-\infty}{0}{\exp{y^2}} 
\] 

Ahora definiremos integrales dentro de un entorno tipo array. El código

\begin{verbatim} 
\begin{align*}
\gdef\integral{\Integral[y]{-\infty}{0}{\exp{y^2}}}
x & = y \integral \\
\gdef\integral{\Integral[z]{-\infty}{0}{\exp{z^2}}} 
\intertext{ redefines y = z} \\
&= z \integral 
\end{align*}

Fuera del align 
\[ \integral \]

redefines $x=y$ :

\def\integral{\Integral[y]{-\infty}{0}{\exp{y^2}}}

\[ \integral \]
\end{verbatim} 


Produce
\begin{align*}%
\gdef\integral{\Integral[y]{-\infty}{0}{\exp{y^2}}}%
x & = y \integral \\ %
\gdef\integral{\Integral[z]{-\infty}{0}{\exp{z^2}}} %
\intertext{ redefines $y = z$}%
&= z \integral%
\end{align*}


Fuera del alingn 
\[ \integral \]

redefines $z=y$ :

\def\integral{\Integral[y]{-\infty}{0}{\exp{y^2}}}

\[ \integral \]


Ahors utilizaremos el paquete \verb+xparse+ veamos el siguiente codigo:

\begin{verbatim}
\DeclareDocumentCommand{\integral}{ o o m D(){x} }{
  \IfNoValueTF{#2}{
    \IfNoValueTF{#1}{
      \int #3 \, d{#4}
    }{
      \int_{#1} #3 \, d{#4}
    }
  }{
    \int_{#1}^{#2} #3 \, d{#4}
 } 
} 

\[
  \Integral[\gamma]{\Integral{\Integral[c][d]{ax}(y)}}(z)
\]
\end{verbatim}

\DeclareDocumentCommand{\integral}{ o o m D(){x} }{
  \IfNoValueTF{#2}{
    \IfNoValueTF{#1}{
      \int #3 \, d{#4}
    }{
      \int_{#1} #3 \, d{#4}
    }
  }{
    \int_{#1}^{#2} #3 \, d{#4}
  } 
 } 

Produce 

\[
  \integral[\gamma]{\integral{\integral[c][d]{ax}(y)}}(z)
\]

\end{document}
